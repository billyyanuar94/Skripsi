\chapter{The Source Code}
\label{app:B}

%selalu gunakan single spacing untuk source code !!!!!
\singlespacing 
% language: bahasa dari kode program
% terdapat beberapa pilihan : Java, C, C++, PHP, Matlab, R, dll
%
% basicstyle : ukuran font untuk kode program
% terdapat beberapa pilihan : tiny, scriptsize, footnotesize, dll
%
% caption : nama yang akan ditampilkan di dokumen akhir, lihat contoh
\begin{lstlisting}[language=PHP,basicstyle=\tiny,caption=C\_skripsi.php]
<?php
defined('BASEPATH') OR exit('No direct script access allowed');

class C_skripsi extends CI_Controller {

/**
* Index Page for this controller.
*
* Maps to the following URL
*              http://example.com/index.php/welcome
*      - or -
*              http://example.com/index.php/welcome/index
*      - or -
* Since this controller is set as the default controller in
* config/routes.php, it's displayed at http://example.com/
*
* So any other public methods not prefixed with an underscore will
* map to /index.php/welcome/<method_name>
* @see https://codeigniter.com/user_guide/general/urls.html
*/
	public function index()
	{
		$this->load->view('skripsi');
	}
	//Check database
	public function view_cekMahasiswa(){
		$data = $this->skripsi_model->getAllMahasiswa();
		$this->load->view('cek_mahasiswa', array('data' => $data));
	}
	
	public function tambahDataMahasiswa(){
		$semester = $_POST['semester'];
		$tahun = $_POST['tahun'];
		$npm = $_POST['npm'];
		$nama = $_POST['nama'];
		$judul = $_POST['judul'];
		$namaPembimbing = $_POST['namaPembimbing'];
		$namaPembimbingPendamping = $_POST['namaPembimbingPendamping'];
		$namaKetuaTimPenguji = $_POST['namaKetuaTimPenguji'];
		$namaAnggotaTimPenguji = $_POST['namaAnggotaTimPenguji'];
		$bobotKetuaTimPenguji = $_POST['bobotKetuaTimPenguji'];
		$bobotAnggotaTimPenguji = $_POST['bobotAnggotaTimPenguji'];
		$bobotPembimbing = $_POST['bobotPembimbing'];
		$nilaiKoordinatorSkripsi = $_POST['nilaiKoordinatorSkripsi'];
		$bobotKoordinatorSkripsi = $_POST['bobotKoordinatorSkripsi'];
		$bobotTataTulisLaporanAnggota = $_POST['bobotTataTulisLaporanAnggota'];
		$bobotKelengkapanMateriAnggota = $_POST['bobotKelengkapanMateriAnggota'];
		$bobotPenguasaanMateriAnggota = $_POST['bobotPenguasaanMateriAnggota'];
		$bobotPresentasiAnggota = $_POST['bobotPresentasiAnggota'];
		$bobotPencapaianTujuanAnggota = $_POST['bobotPencapaianTujuanAnggota'];
		$bobotTataTulisLaporanKetua = $_POST['bobotTataTulisLaporanKetua'];
		$bobotKelengkapanMateriKetua = $_POST['bobotKelengkapanMateriKetua'];
		$bobotPenguasaanMateriKetua = $_POST['bobotPenguasaanMateriKetua'];
		$bobotPresentasiKetua = $_POST['bobotPresentasiKetua'];
		$bobotPencapaianTujuanKetua = $_POST['bobotPencapaianTujuanKetua'];
		$bobotTataTulisLaporanPembimbing = $_POST['bobotTataTulisLaporanPembimbing'];
		$bobotKelengkapanMateriPembimbing = $_POST['bobotKelengkapanMateriPembimbing'];
		$bobotPenguasaanMateriPembimbing = $_POST['bobotPenguasaanMateriPembimbing'];
		$prosesBimbinganPembimbing = $_POST['prosesBimbinganPembimbing'];
		$nilaiAkhirMahasiswa = $_POST['nilaiAkhirMahasiswa'];
		$data_insert = array(
			'semester' => $semester,
			'tahun' => $tahun,
			'npm' => $npm,
			'nama' => $nama,
			'judul' => $judul,
			'namaPembimbing' => $namaPembimbing,
			'namaPembimbingPendamping' => $namaPembimbingPendamping,
			'namaKetuaTimPenguji' => $namaKetuaTimPenguji,
			'namaAnggotaTimPenguji' => $namaAnggotaTimPenguji,
			'bobotKetuaTimPenguji' => $bobotKetuaTimPenguji,
			'bobotAnggotaTimPenguji' => $bobotAnggotaTimPenguji,
			'bobotPembimbing' => $bobotPembimbing,
			'nilaiKoordinatorSkripsi' => $nilaiKoordinatorSkripsi,
			'bobotKoordinatorSkripsi' =>$bobotKoordinatorSkripsi,
			'bobotTataTulisLaporanAnggota' => $bobotTataTulisLaporanAnggota,
			'bobotKelengkapanMateriAnggota' => $bobotKelengkapanMateriAnggota,
			'bobotPenguasaanMateriAnggota' => $bobotPenguasaanMateriAnggota,
			'bobotPresentasiAnggota' => $bobotPresentasiAnggota,
			'bobotPencapaianTujuanAnggota' => $bobotPencapaianTujuanAnggota,
			'bobotTataTulisLaporanKetua' => $bobotTataTulisLaporanKetua,
			'bobotKelengkapanMateriKetua' => $bobotKelengkapanMateriKetua,
			'bobotPenguasaanMateriKetua' => $bobotPenguasaanMateriKetua,
			'bobotPresentasiKetua' => $bobotPresentasiKetua,
			'bobotPencapaianTujuanKetua' => $bobotPencapaianTujuanKetua,
			'bobotTataTulisLaporanPembimbing' => $bobotTataTulisLaporanPembimbing,
			'bobotKelengkapanMateriPembimbing' => $bobotKelengkapanMateriPembimbing,
			'bobotPenguasaanMateriPembimbing' => $bobotPenguasaanMateriPembimbing,
			'prosesBimbinganPembimbing' => $prosesBimbinganPembimbing,
			'nilaiAkhirMahasiswa' => $nilaiAkhirMahasiswa,
		
		);
		$res = $this->skripsi_model->insertDataMahasiswa('beritaacarasidangskripsi',$data_insert);
		redirect(base_url(), 'refresh');
	}

}

\end{lstlisting}

\begin{lstlisting}[language=PHP,basicstyle=\tiny,caption=skripsi.php]
<!DOCTYPE html>
<!--
To change this license header, choose License Headers in Project Properties.
To change this template file, choose Tools | Templates
and open the template in the editor.
-->
<html>
	<head>
		<meta charset="utf-8">
		<meta http-equiv="X-UA-Compatible" content="IE=edge">
		<meta name="viewport" content="width=device-width, initial-scale=1">
		<title> Berita Acara Sidang Skripsi </title>
		
		<!-- Bootstrap Core CSS -->
		<link href="public/css/bootstrap/bootstrap.min.css" rel="stylesheet">
		
		<!-- Custom Scroll Nav CSS -->
		<link href="public/css/scrolling-nav.css" rel="stylesheet">
		
		<!-- Custom CSS -->
		<link href="public/css/custom.css" rel="stylesheet">
		
		<!-- AngularJS -->
		<script src="public/js/angularJS/angular.min.js"></script>
		
		<!-- Mobile friendly bootstrap -->
		
		
	</head>
	<body ng-app="penilaian" id="page-top" data-spy="scroll" data-target=".navbar-fixed-top">
	
	
		<!-- Navigation -->
		<nav class="navbar navbar-default navbar-fixed-top" role="navigation">
			<div class="container">
				<div class="navbar-header page-scroll">
					<button type="button" class="navbar-toggle" data-toggle="collapse" data-target=".navbar-ex1-collapse">
						<span class="sr-only">Toggle navigation</span>
						<span class="icon-bar"></span>
						<span class="icon-bar"></span>
						<span class="icon-bar"></span>
					</button>
				</div>
				
				<form role="form" method="post" accept-charset="utf-8" action="<?php echo base_url() . "index.php/c_skripsi/tambahDataMahasiswa" ?>" ng-controller="DefaultValue">
					
					
					<!-- Collect the nav links, forms, and other content for toggling -->
					<div class="collapse navbar-collapse navbar-ex1-collapse">
						<ul class="nav navbar-nav">
							<!-- Hidden li included to remove active class from about link when scrolled up past about section -->
							<li>
								<a class="page-scroll" href="#page-top">Berita Acara Sidang Skripsi</a>
							</li>
							<li>
								<a class="page-scroll" href="#rekAnggota">Lembar Rekapitulasi Anggota Tim Penguji</a>
							</li>
							<li>
								<a class="page-scroll" href="#rekKetua">Lembar Rekapitulasi Ketua Tim Penguji</a>
							</li>
							<li>
								<a class="page-scroll" href="#rekPembimbing">Lembar Rekapitulasi Pembimbing</a>
							</li>
							<li>
								<a class="page-scroll" href="#selesai">Selesai</a>
							</li>
						</ul>
					</div>
					<!-- /.navbar-collapse -->
				</div>
				<!-- /.container -->
			</nav>
			
			
			
			<!-- Berita Acara Sidang Skripsi -->
			<section id="intro" class="intro-section">
				<!-- Page Heading -->
				<div class="container">
					<div class="row">
						<div class="col-lg-12">
							<div class="page-header">
								<h1>
									Berita Acara Sidang Skripsi
								</h1>
							
								<div class="semester">
									<p> 
									<label>Semester:</label>
									<!-- 1 -->
									<select name="semester">
										<option value="1">Ganjil</option>
										<option value="2">Genap</option>
									</select>
									<!-- 2 -->
									<input id="tahun" type="number" max="9999" ng-model="tahun" name="tahun"/>
									/
									<input id="tahun_1" type="number" max="9999" value="{{tahun + 1}}" disabled="disabled"/>
									
									</p>
								</div>
							</div>
						</div>
					</div>
				<!-- Isi -->
				<div class="row">
					<div class="col-lg-12">
					
						<div class="form-group">
						Telah diselenggarakan Sidang Skripsi untuk mata kuliah AIF402-6 Skripsi 2 bagi:
						
							<div id="pengenalMahasiswa">
								<p>
								<!-- 3 -->
								<label class="col-md-1 col-xs-6" for="npm">NPM:</label><input maxlength="10" id="npm" class="inline-form col-md-3 col-xs-6" ng-model="n_npm" name="npm"/>
								<!-- 4 -->
								<label class="col-md-1 col-xs-6" for="nama">Nama:</label><input id="nama" class="inline-form col-md-7 col-xs-6" name="nama"/>
								</p>
							</div>
							<br/>
							<div id="pengenalJudul">
								<p>
								<!-- 5 -->
								<label class="col-md-1 col-xs-6" for="judul">Judul:</label><input id="judul" class="inline-form col-md-11 col-xs-6" name="judul"/>
								</p>
							</div>
						</div>
					
						<p> dengan pembimbing dan penguji:</p>
					
					
						<div id="pengenalPembimbing">
							<p>
							<label class="col-md-3 col-xs-8" for="pembimbing">Pembimbing:</label>
							<!-- 6 -->
							<input class="col-md-9 col-xs-4" id="pembimbing" name="namaPembimbing"/>
							</p>
						</div>
						<br/>
					
						<div id="pengenalPembimbingPendamping">
							<p>
							<label class="col-md-3 col-xs-8" for="pembimbing2">Pembimbing Pendamping:</label>
							<!-- 7 -->
							<input class="col-md-9 col-xs-4" id="pembimbing2" name="namaPembimbingPendamping"/>
							</p>
						</div>
						<br/>
						<div id="pengenalKetua">
							<p>
							<label class="col-md-3 col-xs-8" for="ketua">Ketua Tim Penguji:</label>
							<!-- 8 -->
							<input class="col-md-9 col-xs-4" id="ketua" name="namaKetuaTimPenguji"/>
							
							</p>
						</div>
						<br/>
					
						<div id="pengenalAnggota">
							<p>
							<label class="col-md-3 col-xs-8" for="anggota">Anggota Tim Penguji:</label>
							<!-- 9 -->
							<input class="col-md-9 col-xs-4" id="anggota" name="namaAnggotaTimPenguji" />
							
							</p>
						</div>
						<br/>
						<p>Rekapitulasi nilai Sidang Skripsi 2 yang diberikan oleh pembimbing, penguji & koordinator skripsi:</p>
						<table class="col-md-8 col-xs-12 col-md-offset-4 col-md-pull-2 table-responsive">
							<tr>
								<th>No</th>
								<th>Pembimbing/Penguji</th>
								<th>Nilai</th>
								<th>Bobot(%)</th>
								<th>Nilai Akhir</th>
							</tr>
							<tr>
								<td>1</td>
								<td><label for="nKetua">Ketua Tim Penguji</label></td>
								<td><input type="number" id="nKetua" max="100" ng-model="nilai_ketua" class="form-nilai" value="{{nilai_TTLaporanK * TTLaporanK.value / 100 + nilai_KMateriK * KMateriK.value / 100 + nilai_PMateriK * PMateriK.value / 100 + nilai_PresentasiK * presentasiK.value / 100 + nilai_PTujuanK * PTujuanK.value / 100}}" disabled="disabled" /></td>
								<!-- 10 -->
								<td><input type="number" ng-model="ketua.value" ng-init="ketua.value = 35" min="0" max="100" class="form-nilai" name="bobotKetuaTimPenguji" readonly="readonly" /></td>
								<td><input type="number" value="{{(nilai_TTLaporanK * TTLaporanK.value / 100 + nilai_KMateriK * KMateriK.value / 100 + nilai_PMateriK * PMateriK.value / 100 + nilai_PresentasiK * presentasiK.value / 100 + nilai_PTujuanK * PTujuanK.value / 100) * ketua.value / 100}}" ng-model="total_ketua" class="form-nilai" disabled="disabled" /></td>
							</tr>
							<tr>
								<td>2</td>
								<td><label for="nAnggota">Anggota Tim Penguji</label></td>
								<td><input id="nAnggota" type="number" max="100" ng-model="nilai_anggota" class="form-nilai" value="{{nilai_TTLaporanA * TTLaporanA.value / 100 + nilai_KMateriA * KMateriA.value / 100 + nilai_PMateriA * PMateriA.value / 100 + nilai_PresentasiA * presentasiA.value / 100 + nilai_PTujuanA * PTujuanA.value / 100}}" disabled="disabled" /></td>
								<!-- 11 -->
								<td><input type="number" ng-model="anggota.value" ng-init="anggota.value = 35" min="0" max="100" class="form-nilai" name="bobotAnggotaTimPenguji" readonly="readonly" /></td>
								<td><input type="number" value="{{(nilai_TTLaporanA * TTLaporanA.value / 100 + nilai_KMateriA * KMateriA.value / 100 + nilai_PMateriA * PMateriA.value / 100 + nilai_PresentasiA * presentasiA.value / 100 + nilai_PTujuanA * PTujuanA.value / 100) * anggota.value / 100}}" ng-model="total_anggota" class="form-nilai" disabled="disabled" /></td>
							</tr>
							<tr>
								<td>3</td>
								<td><label for="nPembimbing">Pembimbing</label></td>
								<td><input id="nPembimbing" type="number" max="100" ng-model="nilai_pembimbing" class="form-nilai" min=0 value="{{nilai_TTLaporanP * TTLaporanP.value / 100 + nilai_KMateriP * KMateriP.value / 100 + nilai_PMateriP * PMateriP.value / 100 + nilai_PBimbinganP * PBimbinganP.value / 100}}" disabled="disabled" /></td>
								<!-- 12 -->
								<td><input type="number" ng-model="pembimbing.value" ng-init="pembimbing.value = 20" min="0" max="100" class="form-nilai" name="bobotPembimbing" readonly="readonly" /></td>
								<td><input type="number" value="{{(nilai_TTLaporanP * TTLaporanP.value / 100 + nilai_KMateriP * KMateriP.value / 100 + nilai_PMateriP * PMateriP.value / 100 + nilai_PBimbinganP * PBimbinganP.value / 100) * pembimbing.value / 100}}" ng-model="total_pembimbing" class="form-nilai" disabled="disabled" /></td>
							</tr>
							<tr>
								<td>4</td>
								<td><label for="nKoordinator">Koordinator Skripsi</label></td>
								<!-- 13 -->
								<td><input id="nKoordinator" type="number" max="100" ng-model="nilai_koordinator" class="form-nilai" min=0 name="nilaiKoordinatorSkripsi"/></td>
								<!-- 14 -->
								<td><input type="number"  ng-model="koordinator.value" ng-init="koordinator.value = 10" min="0" max="100" class="form-nilai" name="bobotKoordinatorSkripsi" readonly="readonly" /></td>
								<td><input type="number" value={{nilai_koordinator*koordinator.value/100}} ng-model="total_koodinator" class="form-nilai" disabled="disabled" /></td>
							</tr>
							<tr>
								<td></td>
								<td colspan="2" ><label for="nTotal">Total</label></td>
								<td><input type="number" id="nTotal" max="100" disabled="disabled" value={{ketua.value+anggota.value+pembimbing.value+koordinator.value}} class="form-nilai"/></td>
								<!-- 29 -->
								<td><input type="number" name="nilaiAkhirMahasiswa" value= "{{(nilai_TTLaporanK * TTLaporanK.value / 100 + nilai_KMateriK * KMateriK.value / 100 + nilai_PMateriK * PMateriK.value / 100 + nilai_PresentasiK * presentasiK.value / 100 + nilai_PTujuanK * PTujuanK.value / 100 )* ketua.value / 100 + (nilai_TTLaporanA * TTLaporanA.value / 100 + nilai_KMateriA * KMateriA.value / 100 + nilai_PMateriA * PMateriA.value / 100 + nilai_PresentasiA * presentasiA.value / 100 + nilai_PTujuanA * PTujuanA.value / 100) * anggota.value / 100 + (nilai_TTLaporanP * TTLaporanP.value / 100 + nilai_KMateriP * KMateriP.value / 100 + nilai_PMateriP * PMateriP.value / 100 + nilai_PBimbinganP * PBimbinganP.value / 100) * pembimbing.value / 100 + nilai_koordinator * koordinator.value / 100}}" class="form-nilai"/></td>
							</tr>
						</table>
					</div> 
				</div>
			</section>
			
			
			
			
			<!-- Rekapitulasi Ketua Tim Penguji -->
			<section id="rekKetua" class="rekKetua-section">
				<!-- Page Heading -->
				<div class="container">
					<div class="row">
						<div class="col-lg-6.col-lg-offset-3">
							<div class="page-header">
								<h1>
									Rekapitulasi Penilaian Skripsi 2 (Ketua Tim Penguji)
								</h1>
								<div class="semester">
									<p> 
									<label for="npmK">NPM:</label><input id="nmpK" maxlength="10" value="{{ n_npm}}" disabled="disabled" />
									</p>
								</div>
							</div>
						</div>
					</div>
					<!-- Isi Rekapitulasi Ketua Tim Penguji -->
					<div class="row">
						<div class="col-lg-12">
							<table class="col-md-8 col-xs-12 col-md-offset-4 col-md-pull-2 table-responsive">
								<tr>
									<th>Komponen Penilaian</th>
									<th>Nilai</th>
									<th>Bobot(%)</th> 
									<th>Nilai Akhir</th>
								</tr>
								<tr>
									<td><label for="nTTLaporanK">Tata Tulis Laporan</label></td>
									<td><input type="number" id="nTTLaporanK" max="100" ng-model="nilai_TTLaporanK" class="form-nilai"/></td>
									<!-- 20 -->
									<td><input type="number" name="bobotTataTulisLaporanKetua" ng-model="TTLaporanK.value" ng-init="TTLaporanK.value = 15" min="0" max="100" class="form-nilai" readonly="readonly" /></td>
									<td><input type="number" disabled="disabled" value="{{nilai_TTLaporanK * TTLaporanK.value / 100}}" ng-model="total_TTLaporanK" class="form-nilai"/></td>
								</tr>
								<tr>
									<td><label for="nKMateriK">Kelengkapan Materi</label></td>
									<td><input type="number" id="nKMateriK" max="100" ng-model="nilai_KMateriK" class="form-nilai"/></td>
									<!-- 21 -->
									<td><input type="number" name="bobotKelengkapanMateriKetua" ng-model="KMateriK.value" ng-init="KMateriK.value = 10" min="0" max="100" class="form-nilai" readonly="readonly" /></td>
									<td><input type="number" disabled="disabled" value="{{nilai_KMateriK * KMateriK.value / 100}}" ng-model="total_KMateriK" class="form-nilai"/></td>
								</tr>
								<tr>
									<td><label for="nPMateriK">Penguasaan Materi</label></td>
									<td><input type="number" id="nPMateriK" max="100" ng-model="nilai_PMateriK" class="form-nilai"/></td>
									<!-- 22 -->
									<td><input type="number" name="bobotPenguasaanMateriKetua" ng-model="PMateriK.value" ng-init="PMateriK.value = 30" min="0" max="100" class="form-nilai" readonly="readonly" /></td>
									<td><input type="number" disabled="disabled" value="{{nilai_PMateriK * PMateriK.value / 100}}" ng-model="total_PMateriA" class="form-nilai"/></td>
								</tr>
								<tr>
									<td><label for="nPresentasiK">Presentasi</label></td>
									<td><input type="number" id="nPresentasiK" max="100" ng-model="nilai_PresentasiK" class="form-nilai"/></td>
									<!-- 23 -->
									<td><input type="number" name="bobotPresentasiKetua" ng-model="presentasiK.value" ng-init="presentasiK.value = 15" min="0" max="100" class="form-nilai" readonly="readonly" /></td>
									<td><input type="number" disabled="disabled" value="{{nilai_PresentasiK * presentasiK.value / 100}}" ng-model="total_PresentasiK" class="form-nilai"/></td>
								</tr>
								<tr>
									<td><label for="nPTujuanK">Pencapaian Tujuan</label></td>
									<td><input type="number" id="nPTujuanK" max="100" ng-model="nilai_PTujuanK" class="form-nilai"/></td>
									<!-- 24 -->
									<td><input type="number" name="bobotPencapaianTujuanKetua" ng-model="PTujuanK.value" ng-init="PTujuanK.value = 30" min="0" max="100" class="form-nilai" readonly="readonly" /></td>
									<td><input type="number" disabled="disabled" value="{{nilai_PTujuanK * PTujuanK.value / 100}}" ng-model="total_PTujuanK" class="form-nilai"/></td>
								</tr>
								<tr>
									<td colspan="2" ><label for="nTotalBobotK">Total</label></td>
									<td><input type="number" id="nTotalBobotK" max="100" disabled="disabled" value={{TTLaporanK.value+KMateriK.value+PMateriK.value+presentasiK.value+PTujuanK.value}} class="form-nilai"/></td>
									<td><input type="number" id="nTotalKetua" ng-model="nTotalKetua" max="100" value= "{{nilai_TTLaporanK * TTLaporanK.value / 100 + nilai_KMateriK * KMateriK.value / 100 + nilai_PMateriK * PMateriK.value / 100 + nilai_PresentasiK * presentasiK.value / 100 + nilai_PTujuanK * PTujuanK.value / 100}}" class="form-nilai" disabled="disabled" /></td>
								</tr>
							</table>
						</div>
					</div>
				</div>
			</section>
			
			<!-- Rekapitulasi Anggota Tim Penguji -->
			<section id="rekAnggota" class="rekAnggota-section">
				<!-- Page Heading -->
				<div class="container">
					<div class="row">
						<div class="col-lg-6.col-lg-offset-3">
							<div class="page-header">
								<h1>
								Rekapitulasi Penilaian Skripsi 2 (Anggota Tim Penguji)
								</h1>
								<div class="semester">
									<p> 
									<label for="npmA">NPM:</label><input id="nmpA" maxlength="10" value="{{ n_npm}}" disabled="disabled" />
									</p>
								</div>
							</div>
						</div>
					</div>
					<!-- Isi Rekapitulasi Anggota Tim Penguji -->
					<div class="row">
						<div class="col-lg-12">
							<table class="col-md-8 col-xs-12 col-md-offset-4 col-md-pull-2 table-responsive">
								<tr>
									<th>Komponen Penilaian</th>
									<th>Nilai</th>
									<th>Bobot(%)</th>
									<th>Nilai Akhir</th>
								</tr>
								<tr>
									<td><label for="nTTLaporanA">Tata Tulis Laporan</label></td>
									<td><input type="number" id="nTTLaporanA" max="100" ng-model="nilai_TTLaporanA" class="form-nilai"/></td>
									<!-- 15 -->
									<td><input type="number" name="bobotTataTulisLaporanAnggota" ng-model="TTLaporanA.value" ng-init="TTLaporanA.value = 15" min="0" max="100" class="form-nilai" readonly="readonly" /></td>
									<td><input type="number" disabled="disabled" value="{{nilai_TTLaporanA * TTLaporanA.value / 100}}" ng-model="total_TTLaporanA" class="form-nilai"/></td>
								</tr>
								<tr>
									<td><label for="nKMateriA">Kelengkapan Materi</label></td>
									<td><input type="number" id="nKMateriA" max="100" ng-model="nilai_KMateriA" class="form-nilai"/></td>
									<!-- 16 -->
									<td><input type="number" name="bobotKelengkapanMateriAnggota" ng-model="KMateriA.value" ng-init="KMateriA.value = 10" min="0" max="100" class="form-nilai" readonly="readonly" /></td>
									<td><input type="number" disabled="disabled" value="{{nilai_KMateriA * KMateriA.value / 100}}" ng-model="total_KMateriA" class="form-nilai"/></td>
								</tr>
								<tr>
									<td><label for="nPMateriA">Penguasaan Materi</label></td>
									<td><input type="number" id="nPMateriA" max="100" ng-model="nilai_PMateriA" class="form-nilai"/></td>
									<!-- 17 -->
									<td><input type="number" name="bobotPenguasaanMateriAnggota" ng-model="PMateriA.value"  ng-init="PMateriA.value = 30" min="0" max="100" class="form-nilai" readonly="readonly" /></td>
									<td><input type="number" disabled="disabled" value="{{nilai_PMateriA * PMateriA.value / 100}}" ng-model="total_PMateriA" class="form-nilai"/></td>
								</tr>
								<tr>
									<td><label for="nPresentasiA">Presentasi</label></td>
									<td><input type="number" id="nPresentasiA" max="100" ng-model="nilai_PresentasiA" class="form-nilai"/></td>
									<!-- 18 -->
									<td><input type="number" name="bobotPresentasiAnggota" ng-model="presentasiA.value" ng-init="presentasiA.value = 15" min="0" max="100" class="form-nilai" readonly="readonly" /></td>
									<td><input type="number" disabled="disabled" value="{{nilai_PresentasiA * presentasiA.value / 100}}" ng-model="total_PresentasiA" class="form-nilai"/></td>
								</tr>
								<tr>
									<td><label for="nPTujuanA">Pencapaian Tujuan</label></td>
									<td><input type="number" id="nPTujuanA" max="100" ng-model="nilai_PTujuanA" class="form-nilai"/></td>
									<!-- 19 -->
									<td><input type="number" name="bobotPencapaianTujuanAnggota" ng-model="PTujuanA.value" ng-init="PTujuanA.value = 30" min="0" max="100" class="form-nilai" readonly="readonly" /></td>
									<td><input type="number" disabled="disabled" value="{{nilai_PTujuanA * PTujuanA.value / 100}}" ng-model="total_PTujuanA" class="form-nilai"/></td>
								</tr>
								<tr>
									<td colspan="2" ><label for="nTotalBobotA">Total</label></td>
									<td><input type="number" id="nTotalBobotA" max="100" disabled="disabled" value={{TTLaporanA.value+KMateriA.value+PMateriA.value+presentasiA.value+PTujuanA.value}} class="form-nilai"/></td>
									<td><input type="number" id="nTotalAnggota" ng-model="nTotalAnggota" max="100" class="form-nilai" value= "{{nilai_TTLaporanA * TTLaporanA.value / 100 + nilai_KMateriA * KMateriA.value / 100 + nilai_PMateriA * PMateriA.value / 100 + nilai_PresentasiA * presentasiA.value / 100 + nilai_PTujuanA * PTujuanA.value / 100}}" disabled="disabled" /></td>
								</tr>
							</table>
							
						</div>
					</div>
				</div>
			</section>
			
			<!-- Rekapitulasi Pembimbing -->
			<section id="rekPembimbing" class="rekPembimbing-section">
				<!-- Page Heading -->
				<div class="container">
					<div class="row">
						<div class="col-lg-6.col-lg-offset-3">
							<div class="page-header">
							<h1>
							Rekapitulasi Penilaian Skripsi 2 (Pembimbing)
							</h1>
								<div class="semester">
									<p> 
									<label for="npmP">NPM:</label><input id="nmpP" maxlength="10" value="{{ n_npm}}" disabled="disabled" />
									</p>
								</div>
							</div>
						</div>
					</div>
					<!-- Isi Rekapitulasi Pembimbing -->
					<div class="row">
						<div class="col-lg-12">
							<table class="col-md-8 col-xs-12 col-md-offset-4 col-md-pull-2 table-responsive">
								<tr>
									<th>Komponen Penilaian</th>
									<th>Nilai</th>
									<th>Bobot(%)</th>
									<th>Nilai Akhir</th>
								</tr>
								<tr>
									<td><label for="nTTLaporanP">Tata Tulis Laporan</label></td>
									<td><input type="number" id="nTTLaporanP" max="100" ng-model="nilai_TTLaporanP" class="form-nilai"/></td>
									<!-- 25 -->
									<td><input type="number" name="bobotTataTulisLaporanPembimbing" ng-model="TTLaporanP.value" ng-init="TTLaporanP.value = 20" min="0" max="100" class="form-nilai" readonly="readonly" /></td>
									<td><input type="number" disabled="disabled" value="{{nilai_TTLaporanP * TTLaporanP.value / 100}}" ng-model="total_TTLaporanP" class="form-nilai"/></td>
								</tr>
								<tr>
									<td><label for="nKMateriP">Kelengkapan Materi</label></td>
									<td><input type="number" id="nKMateriP" max="100" ng-model="nilai_KMateriP" class="form-nilai"/></td>
									<!-- 26 -->
									<td><input type="number" name="bobotKelengkapanMateriPembimbing" ng-model="KMateriP.value" ng-init="KMateriP.value = 20" min="0" max="100" class="form-nilai" readonly="readonly" /></td>
									<td><input type="number" disabled="disabled" value="{{nilai_KMateriP * KMateriP.value / 100}}" ng-model="total_KMateriP" class="form-nilai"/></td>
								</tr>
								<tr>
									<td><label for="nPMateriP">Penguasaan Materi</label></td>
									<td><input type="number" id="nPMateriP" max="100" ng-model="nilai_PMateriP" class="form-nilai"/></td>
									<!-- 27 -->
									<td><input type="number" name="bobotPenguasaanMateriPembimbing" ng-model="PMateriP.value" ng-init="PMateriP.value = 30" min="0" max="100" class="form-nilai" readonly="readonly" /></td>
									<td><input type="number" disabled="disabled" value="{{nilai_PMateriP * PMateriP.value / 100}}" ng-model="total_PMateriP" class="form-nilai"/></td>
								</tr>
								<tr>
									<td><label for="nPBimbinganP">Proses Bimbingan</label></td>
									<td><input type="number" id="nPBimbinganP" max="100" ng-model="nilai_PBimbinganP" class="form-nilai"/></td>
									<!-- 28 -->
									<td><input type="number" name="prosesBimbinganPembimbing" ng-model="PBimbinganP.value" ng-init="PBimbinganP.value = 30" min="0" max="100" class="form-nilai" readonly="readonly" /></td>
									<td><input type="number" disabled="disabled" value="{{nilai_PBimbinganP * PBimbinganP.value / 100}}" ng-model="total_PBimbinganP" class="form-nilai"/></td>
								</tr>
								<tr>
									<td colspan="2" ><label for="nTotalBobotP">Total</label></td>
									<td><input type="number" id="nTotalBobotP" max="100" disabled="disabled" value={{TTLaporanP.value+KMateriP.value+PMateriP.value+PBimbinganP.value}} class="form-nilai"/></td>
									<td><input type="number" id="nTotalPembimbing" ng-model="nTotalPembimbing" max="100"  value= "{{nilai_TTLaporanP * TTLaporanP.value / 100 + nilai_KMateriP * KMateriP.value / 100 + nilai_PMateriP * PMateriP.value / 100 + nilai_PBimbinganP * PBimbinganP.value / 100}}" class="form-nilai" disabled="disabled" /></td>
								</tr>
							</table>
						</div>
					</div>
				</div>
			</section>
			
			<!-- Selesai -->
			<section id="selesai" class="rekPembimbing-section">
				<!-- Page Heading -->
				<div class="container">
					<div class="row">
						<div class="col-lg-6.col-lg-offset-3">
							<div class="page-header">
								<div class= "tanggal">
								<p>
								Ditetapkan di Bandung, <span id="date"></span>
								</p>
								
								
								<script>
								var months = ['Januari', 'Februari', 'Maret', 'April', 'Mei', 'Juni', 'Juli', 'Agustus', 'September', 'Oktober', 'November', 'Desember'];
								var myDays = ['Minggu', 'Senin', 'Selasa', 'Rabu', 'Kamis', 'Jum&#39;at', 'Sabtu'];
								var date = new Date();
								var day = date.getDate();
								var month = date.getMonth();
								var thisDay = date.getDay(),
								thisDay = myDays[thisDay];
								var yy = date.getYear();
								var year = (yy < 1000) ? yy + 1900 : yy;
								
								
								newdate = thisDay + ', ' + day + ' ' + months[month] + ' ' + year;
								document.getElementById("date").innerHTML = newdate;
								
								newyear = parseInt(year);
								document.getElementById("tahun").value = newyear;
								</script>
								<p><input type="submit" name="submit" value="Selesai"></p>
								</div>
							</div>
						</div>
					</div>
				</div>
			</section>	
		</form>
		<!-- Set Default Value pada ng-model type number -->
		<script>
		angular.module('penilaian', [])
		.controller('DefaultValue', ['$scope', function ($scope) {
		
		}]);
		</script>
		
		<!-- jQuery -->
		<script src="public/js/jQuery/jquery.min.js"></script>
		
		<!-- Bootstrap Core JavaScript -->
		<script src="public/js/bootstrap/bootstrap.min.js"></script>
		
		<!-- Scrolling Nav JavaScript -->
		<script src="public/js/jquery.easing.min.js"></script>
		<script src="public/js/scrolling-nav.js"></script>
		
	</body>
</html>
\end{lstlisting}

\begin{lstlisting}[language=PHP,basicstyle=\tiny,caption=skripsi\_model.php]
<?php
defined('BASEPATH') OR exit('No direct script access allowed');

class Skripsi_model extends CI_Model {

	public function insertDataMahasiswa($tableName, $data){
		$res = $this->db->insert($tableName, $data);
	}
	
	public function getAllMahasiswa(){
		$query = $this->db->get('beritaacarasidangskripsi');
		return $query->result_array();
	}
}

\end{lstlisting}
\begin{lstlisting}[language=PHP,basicstyle=\tiny,caption=scrolling-nav.css]
/*!
* Start Bootstrap - Scrolling Nav HTML Template (http://startbootstrap.com)
* Code licensed under the Apache License v2.0.
* For details, see http://www.apache.org/licenses/LICENSE-2.0.
*/

body {
	width: 100%;
	height: 100%;
}

html {
	width: 100%;
	height: 100%;
}

@media(min-width:767px) {
	.navbar {
	padding: 20px 0;
	-webkit-transition: background .5s ease-in-out,padding .5s ease-in-out;
	-moz-transition: background .5s ease-in-out,padding .5s ease-in-out;
	transition: background .5s ease-in-out,padding .5s ease-in-out;
	}
	
	.top-nav-collapse {
		padding: 0;
	}
}

/* Demo Sections - You can use these as guides or delete them - the scroller will work with any sort of height, fixed, undefined, or percentage based.
The padding is very important to make sure the scrollspy picks up the right area when scrolled to. Adjust the margin and padding of sections and children 
of those sections to manage the look and feel of the site. */

.intro-section {
	height: 100%;
	padding-top: 80px;
	background: #fff;
}
\end{lstlisting}

\begin{lstlisting}[language=PHP,basicstyle=\tiny,caption=custom.css]
/* My Custom CSS */
.form-nilai{
	width: 45px;
}

.page-header{
	text-align: center;
}
tr{
	height:30px;
}
.rekAnggota-section {
	height: 100%;
	padding-top: 80px;
	background: #fff;
}

.rekKetua-section {
	height: 100%;
	padding-top: 80px;
	background: #fff;
}

.rekPembimbing-section {
	height: 100%;
	padding-top: 80px;
	background: #fff;
}
input[disabled="disabled"]{
	background-color: #fff;
}

\end{lstlisting}

\begin{lstlisting}[language=PHP,basicstyle=\tiny,caption=scrolling-nav.js]
//jQuery to collapse the navbar on scroll
$(window).scroll(function() {
	if ($(".navbar").offset().top > 50) {
	$(".navbar-fixed-top").addClass("top-nav-collapse");
	} else {
	$(".navbar-fixed-top").removeClass("top-nav-collapse");
	}
});

//jQuery for page scrolling feature - requires jQuery Easing plugin
$(function() {
	$('a.page-scroll').bind('click', function(event) {
		var $anchor = $(this);
		$('html, body').stop().animate({
		scrollTop: $($anchor.attr('href')).offset().top
		}, 1500, 'easeInOutExpo');
		event.preventDefault();
	});
});

\end{lstlisting}