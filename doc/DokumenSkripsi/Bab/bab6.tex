\chapter{Kesimpulan dan Saran}
\label{chap: kesimpulan}

\section{Kesimpulan}
\label{sec: kesimpulan}
	
	Berdasarkan hasil penelitian yang dilakukan, didapatkan kesimpulan-kesimpulan sebagai berikut:
	\begin{enumerate}
		\item Penilaian skripsi terutama pada skripsi 2 masih menggunakan sistem manual, yaitu penilai mengisi dengan menuliskan nilai dan menghitung nilai akhir dengan alat hitung masing-masing pada lembar penilaian yang diberikan.
		\item Proses penyimpanan Sistem Informasi Penilaian Sidang Skripsi 2 dilakukan dengan menyimpan data diri mahasiswa, nama penilai, nilai bobot, dan nilai akhir yang telah dihitung secara otomatis oleh AngularJS.
		\item AngularJS bekerja dengan mengambil nilai input yang diperlukan dan melakukan perhitungan tanpa diperlukannya pergantian \textit{page} pada sistem penilaian, sehingga \textit{single page application} dapat terlaksana dengan maksimal pada sistem penilaian.
	\end{enumerate}

\section{Saran}
\label{sec: saran}

	Berdasarkan pengujian yang dilakukan, berikut adalah beberapa saran untuk pengembang:
	\begin{itemize}
		\item Menambahkan sistem manajemen nilai skripsi untuk melakukan fungsi \textit{select, update,} dan \textit{delete} karena Sistem Penilaian Sidang Skripsi 2 yang dibuat hanya menangani fungsi \textit{insert} ke \textit{database}.
	\end{itemize}
	