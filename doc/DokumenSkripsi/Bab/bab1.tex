\chapter{Pendahuluan}
\label{chap:pendahuluan}

\section{Latar Belakang}
\label{sec:latarBelakang}

	Program Studi Teknik Informatika di Universitas Katolik Parahyangan memiliki beberapa syarat kelulusan. Salah satunya adalah minimal SKS lulus adalah 144 sks yang terdiri dari mata kuliah wajib, pilihan wajib, dan pilihan. Selain itu Indeks Prestasi Kumulatatif (IPK) minimum  yang diperlukan adalah 2.00 dengan maksimum 14 semester. Salah satu mata kuliah wajib yang harus ditempuh dan lulus adalah Skripsi. Skripsi pada Program Studi Teknik Informatika di Universitas Katolik Parahyangan dibagi menjadi 2 mata kuliah yaitu Skripsi 1 dan Skripsi 2. Sidang pada mata kuliah Skripsi 1 dan Skripsi 2 adalah proses yang harus ditempuh untuk mendapatkan nilai akhir. Sidang dilakukan setelah seluruh persyaratan pada mata kuliah Skripsi 1 atau Skripsi 2 diselesaikan. Tugas akhir ini berfokus pada sidang untuk mata kuliah Skripsi 2.
	
	Pada sidang mata kuliah Skripsi 2, penilaian dilakukan oleh ketua tim penguji dan anggota tim penguji. Penilaian pada sidang skripsi 2 bersifat manual, dimana penilai akan menuliskan nilai yang ingin diberikan serta melakukan perhitungan nilai untuk mendapatkan nilai akhir mahasiswa pada lembar rekapitulasi masing-masing penilai. Lembar rekapitulasi yang telah selesai dihitung kemudian akan di kumpulkan kepada ketua tim penguji untuk dituliskan di lembar berita acara sidang skripsi yang kemudian akan diproses menjadi nilai akhir mahasiswa bersangkutan.
	
	Sifat manual ini mengakibatkan kelalaian manusia dalam melakukan penilaian beberapa kali tidak dapat dihindarkan. Kelalaian manusia yang biasa terjadi contohnya adalah kesalahan perhitungan nilai akhir oleh penilai, kesalahan penulisan nama penilai dan NPM mahasiswa yang bersangkutan, kesalahan penulisan semester atau tahun ajaran saat penilaian skripsi\footnote{berdasarkan diskusi dengan dosen pembimbing}. Untuk mengatasi hal-hal tersebut, diperlukan suatu sistem yang dapat menanggulangi masalah pengisian, kalkulasi perhitungan, dan juga penyimpanan skripsi.
	
	Menurut penjelasan di atas, maka penulis mengusulkan otomatisasi sistem dalam penilaian skripsi yang akan dibangun guna mengurangi kesalahan-kesalahan kecil yang dapat berakibat fatal pada nilai mahasiswa yang bersangkutan. Berdasarkan hal tersebut dibangun tugas akhir otomatisasi sistem penilaian skripsi dengan cara membuat sebuah aplikasi berbasis \textit{web} yaitu Sistem Informasi Penilaian Sidang Skripsi2.
		
	Pada tugas akhir ini, akan dibangun sebuah sistem penilaian yang menanggulangi masalah-masalah tersebut, dengan cara menjadikan beberapa masukan(\textit{input}) otomatis dan juga melakukan eksekusi perhitungan nilai akhir sesuai bobot secara otomatis. Hal ini dianggap akan memudahkan penilai dalam melakukan proses penilaian skripsi, karena penilai tidak perlu lagi repot melakukan perhitungan nilai dan juga mengisi masukan-masukan yang sudah terisi secara otomatis. Terdapat banyak fungsi yang dikerjakan pada tugas akhir ini, seperti fungsi \textit{insert} berfungsi untuk memasukkan nilai ke basis data, \textit{update} berfungsi untuk mengubah nilai pada basis data yang ada, \textit{delete} berfungsi untuk menghapus nilai yang ada pada basis data, dan fungsi lainnya. Tugas akhir ini berfokus pada fungsi \textit{insert}.
	
	Dalam tugas akhir ini penulis memakai \textit{framework} AngularJS yang dimiliki oleh perusahaan \textit{Google}. AngularJS merupakan salah satu \textit{framework} yang paling sering digunakan untuk membuat sebuah aplikasi berbasis \textit{web} dengan konsep \textit{Single Page Application (SPA)}. \textit{Single Page Application} merupakan aplikasi berbasis \textit{web} yang memungkinkan sebuah halaman HTML memiliki konten-konten yang dapat digunakan di halaman tersebut tanpa perlu berganti ke halaman lain.
	
	AngularJS juga bisa diintegrasikan dengan aplikasi yang menggunakan \textit{framework} lain, sehingga sangat berguna dalam pengerjaan aplikasi berbasis \textit{web} terutama pada pengerjaan Sistem Informasi Penilaian Sidang Skripsi2 yang akan dibangun.
	
	Selain AngularJS, sistem usulan ini juga memakai 2 \textit{framework} pendukung yaitu CodeIgniter dan Twitter Bootstrap. CodeIgniter dipakai untuk memudahkan aliran data pada sistem usulan yang dibangun, sedangkan Twitter Bootstrap dipakai untuk mempermudah pengaturan tampilan pada sistem usulan yang akan dibangun.
	
\section{Rumusan Masalah}
\label{sec: rumusanMasalah}

	Berikut adalah susunan permasalahan yang akan dibahas pada tugas akhir ini:
	\begin{enumerate}
		\item Bagaimana sistem penilaian skripsi 2 yang ada pada Program Studi Teknik Informatika di Universitas Katolik Parahyangan?
		\item Bagaimana proses penyimpanan nilai skripsi?
		\item Bagaimana AngularJS bekerja pada eksekusi perhitungan nilai akhir?
	\end{enumerate}

\section{Tujuan}
\label{sec: tujuan}

	Berdasarkan rumusan masalah yang telah dibuat, maka tujuan tugas akhir ini dijelaskan ke dalam poin-poin sebagai berikut:
	\begin{enumerate}
		\item Mempelajari sistem penilaian skripsi pada Program Studi Teknik Informatika di Universitas Katolik Parahyangan
		\item Merancang dan mengimplementasi proses penyimpanan nilai skripsi
		\item Mengimplementasi AngularJS untuk mengeksekusi perhitungan nilai akhir
	\end{enumerate}
	
\section{Batasan Masalah}
\label{sec: batasanMasalah}
	
	Tugas akhir ini memiliki batasan-batasan sebagai berikut:
	
	\begin{enumerate}
		\item Tugas akhir ini hanya dilakukan untuk formulir penilaian mata kuliah Skripsi 2
		\item Tugas akhir ini hanya melakukan fungsi \textit{insert} ke basis data
	\end{enumerate}
	
\section{Metode Penelitian}
\label{sec: metodePenelitian}

Dalam tugas akhir ini, akan dilakukan langkah-langkah berikut:

\begin{enumerate}
	\item Melakukan studi terhadap CodeIgniter, Twitter Bootstrap, dan AngularJS sebagai \textit{framework} yang akan dipakai.
	\item Melakukan perancangan untuk implementasi integrasi sistem tersebut.
	\item Melakukan implementasi dari rancangan yang sudah dilakukan.
	\item Melakukan pengujian pada saat sidang skripsi2 sehingga penilai dapat menguji hasil implementasi tersebut.
	\item Menganalisa dan menarik kesimpulan atas hasil tugas akhir yang telah dilaksanakan.
\end{enumerate}
	
\section{Sistematika Penulisan}
\label{sec: sistematikaPenulisan}

Berikut adalah sistematika penulisan dari dokumen ini:

\begin{itemize}
	\item Bab 1 membahas latar belakang, rumusan masalah, tujuan penulisan, batasan-batasan, serta metode yang digunakan pada tugas akhir ini.
	\item Bab 2 membahas teori-teori yang digunakan dalam tugas akhir ini, yaitu AngularJS, Code Igniter, dan Twitter Bootstrap.
	\item Bab 3 menganalisis sistem kini, beserta perubahan-perubahan yang harus dilakukan.
	\item Bab 4 membahas perancangan yang dilakukan sebelum mengimplementasikan integrasi yang dimaksud, mencakup protokol, basisdata, beserta antarmukanya.
	\item Bab 5 membahas implementasi serta pengujian dari integrasi yang telah dilakukan.
	\item Bab 6 membahas kesimpulan dari keseluruhan tugas akhir ini, serta saran-saran yang dapat diberikan untuk tugas akhir berikutnya.
\end{itemize}