\chapter{Pendahuluan}
\label{chap:pendahuluan}

\section{Latar Belakang}
\label{sec:latarBelakang}

	Skripsi merupakan istilah yang digunakan di Indonesia untuk mengilustrasikan suatu karya tulis ilmiah berupa paparan tulisan hasil penelitian sarjana S1 yang membahas suatu permasalahan/fenomena dalam bidang ilmu tertentu dengan menggunakan kaidah-kaidah yang berlaku.
	
	Saat ini, sistem penilaian skripsi di Universitas Katolik Parahyangan khususnya pada jurusan Teknik Informatika masih mengandalkan otak manusia dan dilakukan secara manual. Kerap kali hal tersebut menyebabkan kesulitan dalam pemberian nilai dan resiko terjadinya kesalahan dalam perhitungan nilai skripsi cukup besar. Selain itu, kelalaian akan pemberi nilai dalam menulis (6 menjadi 0) pun dapat menyebabkan kesalahan yang fatal pada nilai skripsi mahasiswa. 
	
	Menurut penjelasan di atas, maka otomatisasi sistem dalam penilaian skripsi sangat dibutuhkan oleh Universitas guna mengurangi kesalahan - kesalahan kecil yang dapat berakibat fatal pada nilai mahasiswa yang bersangkutan. Berdasarkan hal tersebut dibuatlah penelitian otomatisasi sistem penilaian skripsi dengan cara membuat sebuah aplikasi berbasis web yaitu Sistem informasi Penilaian Skripsi.
	
\section{Rumusan Masalah}
\label{sec: rumusanMasalah}

	Berikut adalah susunan permasalahan yang akan dibahas pada penelitian ini:
	\begin{enumerate}
		\item Bagaimana sistem penilaian skripsi pada jurusan Teknik Informatika Universitas Katolik Parahyangan?
		\item Data apa saja yang diperlukan sebagai acuan dalam penilaian skripsi?
		\item Bagaimana tampilan yang dipakai dalam penilaian skripsi?
	\end{enumerate}

\section{Tujuan}
\label{tujuan}

	Berdasarkan rumusan masalah yang telah dibuat, maka tujuan penelitian ini dijelaskan ke dalam
	poin-poin sebagai berikut:
	\begin{enumerate}
		\item Mengetahui sistem penilaian skripsi pada jurusan Teknik Infromatika Universitas Katolik Parahyangan
		\item Mengetahui data yang diperlukan untuk penilaian skripsi
		\item Memodelkan user interface seperti apa yang sebaiknya digunakan di sistem informasi penilaian skripsi
	\end{enumerate}