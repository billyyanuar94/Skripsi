\chapter{Pendahuluan}
\label{chap:pendahuluan}

\section{Latar Belakang}
\label{sec:latarBelakang}

	Program Studi Teknik Informatika di Universitas Katolik Parahyangan memiliki beberapa syarat kelulusan antara lain minimal SKS yang lulus 144 yang terdiri dari mata kuliah wajib dan pilihan, indeks prestasi minimum adalah 2.00 dengan maksimum 14 semester. Salah satu mata kuliah wajib yang harus ditempuh dan lulus adalah skripsi. Skripsi di Program Studi Teknik Informatika di Universitas Katolik Parahyangan dibagi menjadi 2 mata kuliah yaitu skripsi 1 dan skripsi 2. 
	
	Sistem penilaian sidang skripsi 2 pada Program Studi Teknik Informatika di Universitas Katolik Parahyangan masih bersifat manual dimana penilai mengisi data-data mahasiswa  memberikan nilai untuk mahasiswa pada saat sidang dan juga melakukan penghitungan bobot nilai total. 
	
	Sifat manual ini mengakibatkan kelalaian manusia dalam melakukan penilaian beberapa kali tidak dapat dihindarkan. Kelalaian manusia yang biasa terjadi contohnya adalah kesalahan perhitungan nilai akhir oleh penilai, kesalahan penulisan nama dan NPM mahasiswa yang bersangkutan, kesalahan penulisan semester atau tahun ajaran saat penilaian skripsi\footnote{berdasarkan diskusi dengan dosen pembimbing}. Selain itu, penyimpanan nilai skripsi tergolong sulit karena penyimpanan nilai tidak langsung dibarengi dengan npm mahasiswa yang mengerjakan. Untuk mengatasi hal-hal tersebut, diperlukan suatu sistem yang dapat menanggulangi masalah pengisian, kalkulasi perhitungan, dan juga penyimpanan skripsi.
	
	Menurut penjelasan di atas, maka penulis mengusulkan otomatisasi sistem dalam penilaian skripsi yang akan dibuat guna mengurangi kesalahan - kesalahan kecil yang dapat berakibat fatal pada nilai mahasiswa yang bersangkutan. Berdasarkan hal tersebut dibuatlah penelitian otomatisasi sistem penilaian skripsi dengan cara membuat sebuah aplikasi berbasis web yaitu Sistem informasi Penilaian Sidang Skripsi2.
		
	Pada penelitian ini, akan dibuat sebuah sistem penilaian yang menanggulangi masalah-masalah tersebut, dengan cara menjadikan beberapa masukkan(\textit{input}) otomatis dan juga melakukan eksekusi perhitungan nilai akhir sesuai bobot secara otomatis. Hal ini dianggap akan memudahkan penilai dalam melakukan proses penilaian skripsi, karena penilai tidak perlu lagi repot menghitung dan juga mengisi masukkan-masukkan yang sudah terisi secara otomatis.
	
	Dalam penelitian ini saya memakai \textit{framework} AngularJS yang dimiliki oleh perusahaan \textit{Google}. AngularJS merupakan salah satu \textit{framework} yang paling sering digunakan untuk membuat sebuah aplikasi berbasis \textit{web} dengan konsep \textit{Single Page Application (SPA)}. \textit{Single Page Application} merupakan aplikasi berbasis web yang memungkinkan sebuah halaman HTML memiliki konten-konten yang dapat digunakan di halaman tersebut tanpa perlu berganti ke halaman lain.
	
	AngularJS juga bisa diintegrasikan dengan aplikasi yang menggunakan \textit{framework} lain, sehingga sangat berguna dalam pengerjaan aplikasi berbasis \textit{web} terutama pada pengerjaan Sistem Informasi Penilaian Sidang Skripsi2 yang akan dibuat.
	
\section{Rumusan Masalah}
\label{sec: rumusanMasalah}

	Berikut adalah susunan permasalahan yang akan dibahas pada penelitian ini:
	\begin{enumerate}
		\item Bagaimana sistem penilaian skripsi yang ada pada Program Studi Teknik Informatika di Universitas Katolik Parahyangan?
		\item Bagaimana proses penyimpanan nilai skripsi?
		\item Bagaimana AngularJS bekerja pada eksekusi perhitungan nilai akhir?
	\end{enumerate}

\section{Tujuan}
\label{sec: tujuan}

	Berdasarkan rumusan masalah yang telah dibuat, maka tujuan penelitian ini dijelaskan ke dalam poin-poin sebagai berikut:
	\begin{enumerate}
		\item Mempelajari sistem penilaian skripsi pada Program Studi Teknik Informatika di Universitas Katolik Parahyangan
		\item Merancang dan mengimplementasi proses penyimpanan nilai skripsi
		\item Menentukan dan mengimplementasi AngularJS untuk mengeksekusi perhitungan nilai akhir
	\end{enumerate}
	
\section{Batasan Masalah}
\label{sec: batasanMasalah}
	
	Penelitian ini memiliki batasan-batasan sebagai berikut:
	
	\begin{enumerate}
		\item Penelitian ini hanya dilakukan untuk \textit{form} penilaian mata kuliah skripsi 2
		\item Penelitian ini hanya melakukan kueri \textit{insert} ke basis data
	\end{enumerate}
	
\section{Metode Penelitian}
\label{sec: metodePenelitian}

Dalam penelitian ini, akan dilakukan langkah-langkah berikut:

\begin{enumerate}
	\item Melakukan studi terhadap CodeIgniter, Twitter Bootstrap, dan AngularJS sebagai \textit{framework} yang akan dipakai.
	\item Melakukan perancangan untuk implementasi integrasi sistem tersebut.
	\item Melakukan implementasi dari rancangan yang sudah dilakukan.
	\item Melakukan pengujian pada saat sidang skripsi2 sehingga penilai dapat menguji hasil implementasi tersebut.
	\item Menganalisa dan menarik kesimpulan atas hasil penelitian yang telah dilaksanakan.
\end{enumerate}
	
\section{Sistematika Penulisan}
\label{sec: sistematikaPenulisan}

Berikut adalah sistematika penulisan dari dokumen ini:

\begin{itemize}
	\item Bab 1 membahas latar belakang, rumusan masalah, tujuan penulisan, batasan-batasan, serta metode yang digunakan pada penelitian ini.
	\item Bab 2 membahas teori-teori yang digunakan dalam penelitian ini, yaitu AngularJS, Code Igniter, dan Twitter Bootstrap.
	\item Bab 3 menganalisis sistem kini, beserta perubahan-perubahan yang harus dilakukan.
	\item Bab 4 membahas perancangan yang dilakukan sebelum mengimplementasikan integrasi yang dimaksud, mencakup protokol, basisdata, beserta antarmukanya.
	\item Bab 5 membahas implementasi serta pengujian dari integrasi yang telah dilakukan.
	\item Bab 6 membahas kesimpulan dari keseluruhan penelitian ini, serta saran-saran yang dapat diberikan untuk penelitian berikutnya.
\end{itemize}