\chapter{Analisis}
\label{chap: analisis}
	
\section{Analisis Sistem Usulan}
\label{sec: analisisKode}

Analisis sistem usulan dibagi menjadi beberapa tahap, yaitu analisis \textit{back end}, analisis \textit{front end}, dan analisis \textit{database}. Berikut ini penjelasannya:
	
	\subsection{Analisis Back End}
	\label{sub: backEnd}
	
	Analisis tahap \textit{back end} merupakan analisis pada lapisan data akses dan kode-kode yang bekerja secara tidak terlihat pada suatu aplikasi. Pada sistem informasi penilaian sidang skripsi 2, analisis tahap ini membahas tentang pembuatan kode \textit{model, view, controller} dari \textit{codeigniter}. Berikut ini adalah penjelasan lengkapnya:
	
	\subsubsection{Model}
	\label{subsub: modelCI}
	
	Pada bagian ini akan dijelaskan tentang penggunaan \textit{model} pada \textit{codeigniter}. \textit{Model} mempunyai fungsi untuk membuat sambungan dari aplikasi ke basis data. Pada \textit{codeigniter} pemanggilan \textit{model} dilakukan pada file \textit{controller} dengan menggunakan fungsi khusus \textit{codeigniter} yaitu:
	\begin{lstlisting}
	$data = $this->skripsi_model->getAllMahasiswa();
	\end{lstlisting}
	
	Kode di atas merupakan fungsi dari \textit{codeigniter} yang melakukan pemanggilan terhadap \textit{file model} yang akan dipakai. Pada kasus sistem usulan, nama \textit{file model} yang digunakan adalah "skripsi\_model". \textit{Model} sendiri berisi kode-kode sebagai berikut:
	\begin{lstlisting}
		<?php
		defined('BASEPATH') OR exit('No direct script access allowed');
		
		class Skripsi_model extends CI_Model {
		
			public function insertDataMahasiswa($tableName, $data){
				$res = $this->db->insert($tableName, $data);
			}
		}
	\end{lstlisting}
	
	Berikut adalah \textit{method} yang dimiliki oleh kelas \textit{model}:
	\begin{itemize}
		\item public function insertDataMahasiswa(\$tablename, \$data)\\
		Berfungsi untuk melakukan fungsi \textit{insert} pada basis data.\\
		Parameter: 
		\begin{itemize}
			\item tablename merepresentasikan nama tabel basis data.
			\item data merepresentasikan data dari controller yang sudah diubah dan ingin dimasukkan kedalam basis data.
		\end{itemize}
	\end{itemize}
	
	\subsubsection{Controller}
	\label{subsub: controllerCI}
	
	Pada bagian ini akan dijelaskan tentang kode dan kegunaannya pada kelas \textit{controller}. \textit{Controller} merupakan kelas yang mengatur hubungan antara kelas \textit{model} dan \textit{view} pada \textit{codeigniter}. Dengan memanfaatkan fungsi-fungsi dari \textit{codeigniter}, maka kelas \textit{controller} dapat dipersingkat dan dipermudah dalam pembuatannya. Berikut ini adalah kode pada kelas \textit{C\_Skripsi}:
\begin{lstlisting}
	<?php
	defined('BASEPATH') OR exit('No direct script access allowed');
	
	class C_skripsi extends CI_Controller {
	
	/**
	* Index Page for this controller.
	*
	* Maps to the following URL
	*              http://example.com/index.php/welcome
	*      - or -
	*              http://example.com/index.php/welcome/index
	*      - or -
	* Since this controller is set as the default controller in
	* config/routes.php, it's displayed at http://example.com/
	*
	* So any other public methods not prefixed with an underscore will
	* map to /index.php/welcome/<method_name>
	* @see https://codeigniter.com/user_guide/general/urls.html
	*/
		public function index()
		{
			$this->load->view('skripsi');
		}
		//Check database
		public function view_cekMahasiswa(){
			$data = $this->skripsi_model->getAllMahasiswa();
			$this->load->view('cek_mahasiswa', array('data' => $data));
		}
		
		public function tambahDataMahasiswa(){
			$semester = $_POST['semester'];
			$tahun = $_POST['tahun'];
			$npm = $_POST['npm'];
			$nama = $_POST['nama'];
			$judul = $_POST['judul'];
			$namaPembimbing = $_POST['namaPembimbing'];
			$namaPembimbingPendamping = $_POST['namaPembimbingPendamping'];
			$namaKetuaTimPenguji = $_POST['namaKetuaTimPenguji'];
			$namaAnggotaTimPenguji = $_POST['namaAnggotaTimPenguji'];
			$bobotKetuaTimPenguji = $_POST['bobotKetuaTimPenguji'];
			$bobotAnggotaTimPenguji = $_POST['bobotAnggotaTimPenguji'];
			$bobotPembimbing = $_POST['bobotPembimbing'];
			$nilaiKoordinatorSkripsi = $_POST['nilaiKoordinatorSkripsi'];
			$bobotKoordinatorSkripsi = $_POST['bobotKoordinatorSkripsi'];
			$bobotTataTulisLaporanAnggota = $_POST['bobotTataTulisLaporanAnggota'];
			$bobotKelengkapanMateriAnggota = $_POST['bobotKelengkapanMateriAnggota'];
			$bobotPenguasaanMateriAnggota = $_POST['bobotPenguasaanMateriAnggota'];
			$bobotPresentasiAnggota = $_POST['bobotPresentasiAnggota'];
			$bobotPencapaianTujuanAnggota = $_POST['bobotPencapaianTujuanAnggota'];
			$bobotTataTulisLaporanKetua = $_POST['bobotTataTulisLaporanKetua'];
			$bobotKelengkapanMateriKetua = $_POST['bobotKelengkapanMateriKetua'];
			$bobotPenguasaanMateriKetua = $_POST['bobotPenguasaanMateriKetua'];
			$bobotPresentasiKetua = $_POST['bobotPresentasiKetua'];
			$bobotPencapaianTujuanKetua = $_POST['bobotPencapaianTujuanKetua'];
			$bobotTataTulisLaporanPembimbing = $_POST['bobotTataTulisLaporanPembimbing'];
			$bobotKelengkapanMateriPembimbing = $_POST['bobotKelengkapanMateriPembimbing'];
			$bobotPenguasaanMateriPembimbing = $_POST['bobotPenguasaanMateriPembimbing'];
			$prosesBimbinganPembimbing = $_POST['prosesBimbinganPembimbing'];
			$nilaiAkhirMahasiswa = $_POST['nilaiAkhirMahasiswa'];
			$data_insert = array(
				'semester' => $semester,
				'tahun' => $tahun,
				'npm' => $npm,
				'nama' => $nama,
				'judul' => $judul,
				'namaPembimbing' => $namaPembimbing,
				'namaPembimbingPendamping' => $namaPembimbingPendamping,
				'namaKetuaTimPenguji' => $namaKetuaTimPenguji,
				'namaAnggotaTimPenguji' => $namaAnggotaTimPenguji,
				'bobotKetuaTimPenguji' => $bobotKetuaTimPenguji,
				'bobotAnggotaTimPenguji' => $bobotAnggotaTimPenguji,
				'bobotPembimbing' => $bobotPembimbing,
				'nilaiKoordinatorSkripsi' => $nilaiKoordinatorSkripsi,
				'bobotKoordinatorSkripsi' =>$bobotKoordinatorSkripsi,
				'bobotTataTulisLaporanAnggota' => $bobotTataTulisLaporanAnggota,
				'bobotKelengkapanMateriAnggota' => $bobotKelengkapanMateriAnggota,
				'bobotPenguasaanMateriAnggota' => $bobotPenguasaanMateriAnggota,
				'bobotPresentasiAnggota' => $bobotPresentasiAnggota,
				'bobotPencapaianTujuanAnggota' => $bobotPencapaianTujuanAnggota,
				'bobotTataTulisLaporanKetua' => $bobotTataTulisLaporanKetua,
				'bobotKelengkapanMateriKetua' => $bobotKelengkapanMateriKetua,
				'bobotPenguasaanMateriKetua' => $bobotPenguasaanMateriKetua,
				'bobotPresentasiKetua' => $bobotPresentasiKetua,
				'bobotPencapaianTujuanKetua' => $bobotPencapaianTujuanKetua,
				'bobotTataTulisLaporanPembimbing' => $bobotTataTulisLaporanPembimbing,
				'bobotKelengkapanMateriPembimbing' => $bobotKelengkapanMateriPembimbing,
				'bobotPenguasaanMateriPembimbing' => $bobotPenguasaanMateriPembimbing,
				'prosesBimbinganPembimbing' => $prosesBimbinganPembimbing,
				'nilaiAkhirMahasiswa' => $nilaiAkhirMahasiswa,
			
			);
			$res = $this->skripsi_model->insertDataMahasiswa('beritaacarasidangskripsi',$data_insert);
			redirect(base_url(), 'refresh');
		}
	
	}
	
\end{lstlisting}

Berikut adalah \textit{method-method} yang dimiliki oleh kelas \textit{controller}:
\begin{itemize}
	\item view\_cekMahasiswa\\
	Berfungsi untuk memilih \textit{file view} dan \textit{model} yang akan dipakai pada sistem informasi.
	\item tambahDataMahasiswa\\
	Berfungsi untuk mengambil data yang telah terisi dari \textit{view} sistem dan mengubahnya menjadi \textit{compatible} sehingga dapat diproses kedalam \textit{method} insertDataMahasiswa pada kelas \textit{model} yang kemudian akan diproses ke dalam bahasa sql.
\end{itemize}
	
	\subsubsection{View}
	\label{subsub: viewCI}
	
	Pada bagian ini, akan dijelaskan proses pembuatan \textit{view} pada sistem usulan. \textit{View} dibuat dengan bahasa php yang digabungkan dengan \textit{framework} AngularJS dan bootstrap untuk fungsi tampilan dan otomatisasi perhitungan nilai. \textit{File-file} css dan javascript bootstrap berada pada \textit{folder public}.
	
	\begin{figure}[H]
		\centering
		\includegraphics[scale=0.75]{Gambar/strukturFile}
		\includegraphics[scale=0.75]{Gambar/strukturFilePublic}
		\caption{Struktur File public}
		\label{fig:public}
	\end{figure}
	
	\textit{Views} sendiri dijalankan dengan pemanggilan pada kelas controller seperti yang telah dijelaskan sebelumnya. Hasil pada proses \textit{view} merupakan tampilan yang akan dijelaskan lebih lanjut di \ref{sec: perancanganTampilan}.
	
	\subsection{Analisis Front End}
	\label{sub: frontEnd}
	
	Pada subbab ini akan dijelaskan bagaimana pembuatan dan fungsi otomatisasi dari AngularJS di sistem usulan. Berikut ini adalah contoh proses otomatisasi pada sistem usulan:
	
\begin{lstlisting}
<body ng-app="penilaian" id="page-top" data-spy="scroll" data-target=".navbar-fixed-top">
	<form role="form" method="post" accept-charset="utf-8" action="<?php echo base_url() . "index.php/c_skripsi/tambahDataMahasiswa" ?>" ng-controller="DefaultValue">
	
	</form>
	
	<script>
	angular.module('penilaian', [])
	.controller('DefaultValue', ['$scope', function ($scope) {
	
	}]);
	</script>
</body>
\end{lstlisting}
	
	Contoh diatas adalah inisialisasi dari AngularJS dengan menggunakan fungsi ng-app dan ng-controller yang mengatur keseluruhan fungsi otomatisasi pada sistem usulan. Pada baris pertama dilakukan inisialisasi ng-app yang berfungsi menginisialisasi nama app yang digunakan pada sistem. Setelah ng-app diinisialisasi, baru sistem usulan dapat menggunakan fungsi-fungsi AngularJS seperti menginisialisasi ng-controller pada baris ke-2 dengan nama "Default Value".\\
	Agar \textit{controller} dapat berfungsi, perlu dilakukan pemanggilan terhadap ng-controller dengan memanfaatkan fungsi "angular.module". Baris ke-7 bekerja dengan \textit{parameter} ng-app("penilaian"). Setelah melakukan inisialisasi modul, maka kita dapat memanggil fungsi-fungsi daripada AngularJS untuk dijalankan, seperti \textit{controller("DefaultValue")} ke dalam aplikasi AngularJS.
	
\begin{lstlisting}
<tr>
	<td><label for="nTTLaporanK">Tata Tulis Laporan</label></td>
	<td><input type="number" id="nTTLaporanK" max="100" ng-model="nilai_TTLaporanK" class="form-nilai"/></td>
	<!-- 20 -->
	<td><input type="number" name="bobotTataTulisLaporanKetua" ng-model="TTLaporanK.value" ng-init="TTLaporanK.value = 15" min="0" max="100" class="form-nilai" readonly="readonly" /></td>
	<td><input type="number" disabled="disabled" value="{{nilai_TTLaporanK * TTLaporanK.value / 100}}" ng-model="total_TTLaporanK" class="form-nilai"/></td>
</tr>
\end{lstlisting}
	
	Contoh diatas diambil dari kode \textit{file view} untuk mengatur otomatisasi pada kolom tata tulis laporan milik ketua tim penguji. Pada baris ke-3 dan baris ke-5 adalah contoh kode diatas merupakan contoh inisialisasi fungsi ng-model dari AngularJS, sementar baris ke-6 dilakukan perhitungan otomatis dari ng-model baris ke-3 dikalikan dengan ng-model baris ke-5 dan hasilnya ditampung di nilai \textit{value} yang kemudian akan muncul ke layar \textit{user} secara otomatis.
		
	\subsection{Analisis Database}
	\label{sub: analisisDatabase}
	
	Sistem informasi penilaian sidang skripsi 2 menggunakan perangkat lunak phpmyadmin sebagai sarana penyimpanan dan pengolahan \textit{database}. Seperti yang dijelaskan pada subbab \ref{sec: analisisKode}, didalam \textit{folder} "application" terdapat \textit{folder} "models" (Gambar\ref{fig:strukturFileApp}) yang berfungsi menghubungkan \textit{database} dengan sistem. 
	
	Berdasarkan analisa dari contoh form penilaian skripsi yang ada (gambar \ref{fig: skripsiAsli} dan gambar \ref{fig: rekapAsli}), dapat disimpulkan bahwa penilaian skripsi membutuhkan data-data sebagai berikut:
		
		\begin{enumerate}
			\item Semester
			\item Tahun ajaran
			\item NPM mahasiswa 
			\item Nama mahasiswa
			\item Judul skripsi
			\item Nama pembimbing utama/tunggal
			\item Nama pembimbing pendamping(tidak harus)
			\item Nama ketua tim penguji
			\item Nama anggota tim penguji
			\item Bobot ketua tim penguji
			\item Bobot anggota tim penguji
			\item Bobot pembimbing
			\item Nilai koordinator skripsi
			\item Bobot koordinator skripsi
			\item Bobot tata tulis laporan ketua
			\item Bobot kelengkapan materi ketua
			\item Bobot penguasaan materi ketua
			\item Bobot presentasi ketua
			\item Bobot pencapaian tujuan ketua
			\item Bobot tata tulis laporan anggota
			\item Bobot kelengkapan materi anggota
			\item Bobot penguasaan materi anggota
			\item Bobot presentasi anggota
			\item Bobot pencapaian tujuan anggota
			\item Bobot tata tulis laporan pembimbing
			\item Bobot kelengkapan materi pembimbing
			\item Bobot penguasaan materi pembimbing
			\item Bobot bimbingan pembimbing
			\item Nilai akhir mahasiswa
		\end{enumerate}
	
	Berdasarkan diskusi dengan dosen pembimbing, disimpulkan bahwa sistem penilaian sidang skripsi 2 ini hanya memerlukan penyimpanan untuk bobot masing-masing penilaian dan nilai akhir mahasiswa untuk tahap perhitungan. Hal ini dikarenakan nilai-nilai lainnya dapat dihasilkan dengan melakukan perhitungan pada  nilai akhir mahasiswa dan bobot nilai yang diinginkan. Begitu pula dengan nilai dari masing-masing penguji.
	
\section{Use Case Diagram}
\label{sec: usecaseDiagram}

	\begin{figure}[H]
		\centering
		\includegraphics[scale=0.75]{Gambar/usecase}
		\caption{Use case diagram}
		\label{fig:usecase}
	\end{figure}
	
	\begin{enumerate}
		\item Skenario memasukkan nilai\\
			Deskripsi: Kegiatan memasukkan nilai ke dalam kotak input yang ada.\\
			Aktor: Pengguna\\
			Prakondisi: - \\
			Skenario:
			\begin{itemize}
				\item Pengguna memilih tempat/kolom yang sudah tersedia di tampilan
				\item Pengguna memasukkan nilai yang diinginkan pada tempat/kolom yang telah dipilih.
			\end{itemize}
		\item Skenario melakukan perhitungan otomatis\\
			Deskripsi: Kegiatan melakukan perhitungan secara otomatis pada tampilan\\
			Aktor: Sistem\\
			Prakondisi: Tempat atau kolom nilai yang ingin dihitung sudah terisi\\
			Skenario:
			\begin{itemize}
				\item Pengguna mengisi kolom nilai yang sudah disediakan
				\item Dengan ng-model, sistem mengambil nilai dari tempat/kolom yang sudah diisi dan melakukan perhitungan
				\item Sistem menampilkan hasil perhitungan ke dalam kolom yang disediakan untuk hasil perhitungan.
			\end{itemize}
	\end{enumerate}
	