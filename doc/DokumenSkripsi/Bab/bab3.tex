\chapter{Analisis}
\label{chap: analisis}

\section{Analisis Data Penilaian Skripsi}
\label{sec: analisisData}

	Berdasarkan analisa dari contoh form penilaian skripsi yang ada, dapat disimpulkan bahwa penilaian skripsi membutuhkan data-data sebagai berikut:
		
		\begin{itemize}
			\item Semester
			\item Tahun ajaran
			\item NPM mahasiswa 
			\item Nama mahasiswa
			\item Judul skripsi
			\item Nama Pembimbing utama/tunggal
			\item Nama Pembimbing pendamping(tidak harus)
			\item Nama Ketua tim penguji
			\item Nama Anggota tim penguji
			\item dan bobot masing-masing penilaian
		\end{itemize}
	
	Berdasarkan diskusi dengan dosen pembimbing, disimpulkan bahwa sistem penilaian sidang skripsi 2 ini hanya memerlukan penyimpanan untuk bobot masing-masing penilaian dan nilai akhir mahasiswa untuk tahap perhitungan. Hal ini dikarenakan nilai-nilai lainnya dapat dihasilkan dengan melakukan perhitungan pada  nilai akhir mahasiswa dan bobot nilai yang diinginkan. Begitu pula dengan nilai dari masing-masing penguji.
	
\section{Analisis Tampilan Sistem Informasi Penilaian Skripsi}
\label{sec: analisisTampilan}
	
	Tampilan pada sistem informasi penilaian skripsi haruslah dibuat semirip mungkin dengan form penilaian skripsi yang sudah ada.
		
	\begin{figure}[]
		\centering
		\caption{Perbandingan Tampilan \cite{presentasi}}
		\label{fig:tampilan}
	\end{figure}
	
	Perbedaan yang akan ditampilkan adalah dengan adanya otomatisasi penghitungan nilai sesuai dengan bobot yang diberikan kepada penilai. Hal ini akan memberikan kemudahan penilai untuk melakukan penilaian.
	
	Berikut adalah bayangan awal tampilan untuk sistem informasi penilaian skripsi:
	\begin{figure}[]
		\centering
		\caption{Perkiraan Tampilan \cite{presentasi}}
		\label{fig:tampilan1}
	\end{figure}
	