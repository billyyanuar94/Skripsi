\documentclass[a4paper,twoside]{article}
\usepackage[T1]{fontenc}
\usepackage[bahasa]{babel}
\usepackage{graphicx}
\usepackage{graphics}
\usepackage{float}
\usepackage[cm]{fullpage}
\pagestyle{myheadings}
\usepackage{etoolbox}
\usepackage{setspace} 
\usepackage{lipsum} 
\setlength{\headsep}{30pt}
\usepackage[inner=2cm,outer=2.5cm,top=2.5cm,bottom=2cm]{geometry} %margin
% \pagestyle{empty}

\makeatletter
\renewcommand{\@maketitle} {\begin{center} {\LARGE \textbf{ \textsc{\@title}} \par} \bigskip {\large \textbf{\textsc{\@author}} }\end{center} }
\renewcommand{\thispagestyle}[1]{}
\markright{\textbf{\textsc{AIF401/AIF402 \textemdash Rencana Kerja Skripsi \textemdash Sem. Genap 2015/2016}}}

\onehalfspacing
 
\begin{document}

\title{\@judultopik}
\author{\nama \textendash \@npm} 

%tulis nama dan NPM anda di sini:
\newcommand{\nama}{Billy Yanuar}
\newcommand{\@npm}{2012730017}
\newcommand{\@judultopik}{Sistem Penilaian Sidang Skripsi 2 dengan AngularJS} % Judul/topik anda
\newcommand{\jumpemb}{1} % Jumlah pembimbing, 1 atau 2
\newcommand{\tanggal}{11/02/2016}
\maketitle

\pagenumbering{arabic}

\section{Deskripsi}
Sistem penilaian sidang skripsi 2 pada Program Studi Teknik Informatika di Universitas Katolik Parahyangan masih bersifat manual. Sifat manual ini mengakibatkan kelalaian manusia dalam melakukan penilaian pun beberapa kali tidak dapat dihindarkan. Kelalaian manusia yang biasa terjadi contohnya adalah kesalahan perhitungan nilai akhir oleh penilai, kesalahan penulisan nama dan NPM mahasiswa yang bersangkutan, kesalahan penulisan semester atau tahun ajaran saat penilaian skripsi \footnote{berdasarkan diskusi dengan dosen pembimbing}. Selain itu, penyimpanan nilai skripsi pun tergolong sulit karena tidak langsung dibarengi dengan nilai dan npm mahasiswa yang mengerjakan. Untuk mengatasi hal-hal tersebut, diperlukan suatu sistem yang dapat menanggulangi masalah pengisian, kalkulasi perhitungan, dan juga penyimpanan skripsi.
Pada penelitian ini, akan dibuat sebuah sistem penilaian yang menanggulangi masalah-masalah tersebut dengan cara membuat beberapa masukan dijadikan otomatis dan juga melakukan eksekusi perhitungan nilai akhir sesuai bobot secara otomatis. Hal ini dianggap akan memudahkan penilai dalam proses penilaian skripsi, karena penilai tidak perlu lagi repot menghitung dan juga mengisi hal-hal yang sudah terisi secara otomatis.


\section{Rumusan Masalah}
Rumusan masalah yang dapat disimpulkan dari skripsi ini antara lain:
\begin{itemize}
	\item Bagaimana sistem penilaian skripsi yang ada pada Program Studi Teknik Informatika di Universitas Katolik Parahyangan?
	\item Bagaimana AngularJS bekerja pada eksekusi perhitungan nilai akhir?
	\item Bagaimana proses penyimpanan nilai skripsi?
\end{itemize}

\section{Tujuan}
Tujuan dari skripsi ini antara lain adalah:
\begin{itemize}
	\item Mempelajari sistem penilaian skripsi pada Program Studi Teknik Informatika di Universitas Katolik Parahyangan
	\item Menentukan dan mengimplementasi AngularJS untuk mengeksekusi perhitungan nilai 
	\item Merancang dan mengimplementasi proses penyimpanan nilai skripsi
\end{itemize}

\section{Deskripsi Perangkat Lunak}
Perangkat lunak akhir yang akan dibuat memiliki fitur minimal sebagai berikut:
\begin{itemize}
	\item Menerima masukan berupa data mahasiswa yang sedang melakukan sidang
	\item Menerima masukan berupa nilai yang diberikan oleh pembimbing dan penguji
	\item Melakukan kalkulasi otomatis nilai akhir skripsi
	\item Melakukan pencatatan terhadap nilai skripsi yang dimasukkan
\end{itemize}

\section{Detail Pengerjaan Skripsi}
Bagian-bagian pekerjaan skripsi ini adalah sebagai berikut :
	\begin{enumerate}
		\item Mempelajari lembar penilaian skripsi yang dipakai saat ini
		\item Mempelajari bahasa pemrograman AngularJS, Code Igniter, dan Bootstrap
		\item Merancang tampilan form yang akan dipakai untuk penilaian
		\item Mengimplementasikan fitur fitur yang ada pada form tersebut
		\item Melakukan pengujian (dan eksperimen) yang melibatkan responden untuk menilai hasil simulasi secara kualitatif
		\item Menulis dokumen skripsi
	\end{enumerate}

\section{Rencana Kerja}

Berikut adalah rencana kerja saya di skripsi ini:

\begin{center}
  \begin{tabular}{ | c | c | c | c | l |}
    \hline
    1*  & 2*(\%) & 3*(\%) & 4*(\%) &5*\\ \hline \hline
    1   & 20  & 20  &  &  \\ \hline
    2   & 15 & 5  & 10  & {\footnotesize penulisan dokumen di S2} \\ \hline
    3   & 10  & 10  &  &   \\ \hline
    4   & 15  & 15  &   &  \\ \hline
    5   & 20 &   & 20 &  \\ \hline
    6   & 20 & 	5 & 15  & {\footnotesize penulisan bab 1 pada S1} \\ \hline
    Total  & 100  & 55  & 45 &  \\ \hline
                          \end{tabular}
\end{center}

Keterangan (*)\\
1 : Bagian pengerjaan Skripsi (nomor disesuaikan dengan detail pengerjaan di bagian 5)\\
2 : Persentase total \\
3 : Persentase yang akan diselesaikan di Skripsi 1 \\
4 : Persentase yang akan diselesaikan di Skripsi 2 \\
5 : Penjelasan singkat apa yang dilakukan di S1 (Skripsi 1) atau S2 (skripsi 2)

\pagebreak
\vspace{1cm}
\centering Bandung, \tanggal\\
\vspace{2cm} \nama \\ 
\vspace{1cm}


Menyetujui, \\
\ifdefstring{\jumpemb}{2}{
\vspace{1.5cm}
\begin{centering} Menyetujui,\\ \end{centering} \vspace{0.75cm}
\begin{minipage}[b]{0.45\linewidth}
% \centering Bandung, \makebox[0.5cm]{\hrulefill}/\makebox[0.5cm]{\hrulefill}/2013 \\
\vspace{2cm} Nama: \makebox[3cm]{\hrulefill}\\ Pembimbing Utama
\end{minipage} \hspace{0.5cm}
\begin{minipage}[b]{0.45\linewidth}
% \centering Bandung, \makebox[0.5cm]{\hrulefill}/\makebox[0.5cm]{\hrulefill}/2013\\
\vspace{2cm} Nama: \makebox[3cm]{\hrulefill}\\ Pembimbing Pendamping
\end{minipage}
\vspace{0.5cm}
}{
% \centering Bandung, \makebox[0.5cm]{\hrulefill}/\makebox[0.5cm]{\hrulefill}/2013\\
\vspace{2cm} Nama: \makebox[3cm]{\hrulefill}\\ Pembimbing Tunggal
}

\end{document}

