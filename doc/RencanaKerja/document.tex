\documentclass[a4paper,twoside]{article}
\usepackage[T1]{fontenc}
\usepackage[bahasa]{babel}
\usepackage{graphicx}
\usepackage{graphics}
\usepackage{float}
\usepackage[cm]{fullpage}
\pagestyle{myheadings}
\usepackage{etoolbox}
\usepackage{setspace} 
\usepackage{lipsum} 
\setlength{\headsep}{30pt}
\usepackage[inner=2cm,outer=2.5cm,top=2.5cm,bottom=2cm]{geometry} %margin
% \pagestyle{empty}

\makeatletter
\renewcommand{\@maketitle} {\begin{center} {\LARGE \textbf{ \textsc{\@title}} \par} \bigskip {\large \textbf{\textsc{\@author}} }\end{center} }
\renewcommand{\thispagestyle}[1]{}
\markright{\textbf{\textsc{AIF401/AIF402 \textemdash Rencana Kerja Skripsi \textemdash Sem. Genap 2015/2016}}}

\onehalfspacing
 
\begin{document}

\title{\@judultopik}
\author{\nama \textendash \@npm} 

%tulis nama dan NPM anda di sini:
\newcommand{\nama}{Billy Yanuar}
\newcommand{\@npm}{2012730017}
\newcommand{\@judultopik}{Sistem Operasi Penilaian Skripsi dengan AngularJS} % Judul/topik anda
\newcommand{\jumpemb}{1} % Jumlah pembimbing, 1 atau 2
\newcommand{\tanggal}{11/02/2016}
\maketitle

\pagenumbering{arabic}

\section{Deskripsi}
Sistem penilaian skripsi khususnya pada program studi teknik informatika di Universitas Katolik Parhyangan masih bersifat bersifat manual. Sifat manual ini mengakibatkan kelalaian manusia dalam melakukan penilaian pun beberapa kali tidak dapat dihindarkan. Kelalaian manusia yang biasa terjadi contohnya adalah kesalahan perhitungan nilai akhir oleh penilai, kesalahan penulisan nama dana npm mahasiswa yang bersangkutan, kesalahan penulisan semester atau tahun ajaran saat penilaian skripsi. Selain itu, penyimpanan skripsi pun tergolong sulit karena tidak langsung dibarengi dengan nilai dan npm mahasiswa yang mengerjakan. Untuk mengatasi hal-hal tersebut, diperlukan suatu sistem yang dapat menanggulangi masalah pengisian, kalkulasi perhitungan, dan juga penyimpanan skripsi.
Pada Skripsi ini, akan dibuat sebuah sistem informasi yang menanggulangi masalah-masalah tersebut dengan cara membuat beberapa input dijadikan otomatis dan juga melakukan eksekusi perhitungan nilai akhir sesuai bobot secara otomatis. Hal ini dianggap akan memudahkan penilai dalam proses penilaian skripsi, karena penilai tidka perlu lagi repot menghitung dan juga mengisi hal-hal yang sudah terisi secara otomatis.


\section{Rumusan Masalah}
\begin{itemize}
	\item Bagaimana sistem penilaian skripsi yang ada di Universitas Parahyangan pada program studi Teknik Informatika?
	\item Bagaimana proses data untuk penyimpanan skripsi?
	\item Bagaimana AngularJS bekerja pada eksekusi perhitungan nilai akhir?
\end{itemize}

\section{Tujuan}
Tujuan dari skripsi ini antara lain adalah:
\begin{itemize}
	\item Mempelajari sistem penilaian skripsi di Universitas Parahyangan pada program studi Teknik Informatika
	\item Mempelajari proses data untuk penyimpanan skripsi
	\item Mempelajari AngularJS
\end{itemize}

\section{Deskripsi Perangkat Lunak}
Perangkat lunak akhir yang akan dibuat memiliki fitur minimal sebagai berikut:
\begin{itemize}
	\item Input nama mahasiswa
	\item Input npm mahasiswa
	\item Input judul skripsi
	\item Input pembimbing utama
	\item input pembimbing pendamping
	\item Input ketua tim penguji
	\item Input anggota tim penguji
	\item Pengguna dapat mengisi nilai yang ingin diberikan di kolom yang sesuai dan sistem akan memunculkan nilai akhir sesuai dengan presentase yang ada secara otomatis
	\item Otomatisasi semester dan tahun ajaran
\end{itemize}

\section{Detail Pengerjaan Skripsi}
Tuliskan bagian-bagian pengerjaan skripsi secara detail. Bagian pekerjaan tersebut mencakup awal hingga akhir skripsi, termasuk di dalamnya pengerjaan dokumentasi skripsi, pengujian, survei, dll.

Bagian-bagian pekerjaan skripsi ini adalah sebagai berikut :
	\begin{enumerate}
		\item Mempelajari github sebagai basis penyimpanan skripsi
		\item Mempelajari lembar penilaian skripsi yang dipakai saat ini
		\item Mempelajari input apa saja yang diperlukan dalam form skripsi
		\item Mempelajari bahasa pemrograman AngularJS
		\item Merancang tampilan form yang akan dipakai untuk penilaian
		\item Mengimplementasikan fitur fitur yang ada pada form tersebut
		\item Melakukan pengujian (dan eksperimen) yang melibatkan responden untuk menilai hasil simulasi secara kualitatif
		\item Menulis dokumen skripsi
	\end{enumerate}

\section{Rencana Kerja}
Tuliskan rencana anda untuk menyelesaikan skripsi. Rencana kerja dibagi menjadi dua bagian yaitu yang akan dilakukan pada saat mengambil kuliah AIF401 Skripsi 1 dan pada saat mengambil kuliah AIF402 Skripsi 2. Perhatikan contoh berikut ini :


\begin{center}
  \begin{tabular}{ | c | c | c | c | l |}
    \hline
    1*  & 2*(\%) & 3*(\%) & 4*(\%) &5*\\ \hline \hline
    1   & 5  & 5  &  &  \\ \hline
    2   & 5 & 5  &   & \\ \hline
    3   & 10  & 9  & 1 & {\footnotesize melengkapi kembali input yang diperlukan di s2}  \\ \hline
    4   & 15  & 6  &  9 & {\footnotesize teknik menyimpan dan kalkulasi di S2} \\ \hline
    5   & 20 & 5  & 15 & {\footnotesize perancangan awal prototipe di S1 dan di tingkatkan di S2} \\ \hline
    6   & 20 &   & 20  & \\ \hline
    7   & 20  & 5  & 15 &  {\footnotesize eksperimen prototipe di S1}\\ \hline
    8   & 5  &   &  5  & \\ \hline
    Total  & 100  & 40  & 60 &  \\ \hline
                          \end{tabular}
\end{center}

Keterangan (*)\\
1 : Bagian pengerjaan Skripsi (nomor disesuaikan dengan detail pengerjaan di bagian 5)\\
2 : Persentase total \\
3 : Persentase yang akan diselesaikan di Skripsi 1 \\
4 : Persentase yang akan diselesaikan di Skripsi 2 \\
5 : Penjelasan singkat apa yang dilakukan di S1 (Skripsi 1) atau S2 (skripsi 2)

\vspace{1cm}
\centering Bandung, \tanggal\\
\vspace{2cm} \nama \\ 
\vspace{1cm}

Menyetujui, \\
\ifdefstring{\jumpemb}{2}{
\vspace{1.5cm}
\begin{centering} Menyetujui,\\ \end{centering} \vspace{0.75cm}
\begin{minipage}[b]{0.45\linewidth}
% \centering Bandung, \makebox[0.5cm]{\hrulefill}/\makebox[0.5cm]{\hrulefill}/2013 \\
\vspace{2cm} Nama: \makebox[3cm]{\hrulefill}\\ Pembimbing Utama
\end{minipage} \hspace{0.5cm}
\begin{minipage}[b]{0.45\linewidth}
% \centering Bandung, \makebox[0.5cm]{\hrulefill}/\makebox[0.5cm]{\hrulefill}/2013\\
\vspace{2cm} Nama: \makebox[3cm]{\hrulefill}\\ Pembimbing Pendamping
\end{minipage}
\vspace{0.5cm}
}{
% \centering Bandung, \makebox[0.5cm]{\hrulefill}/\makebox[0.5cm]{\hrulefill}/2013\\
\vspace{2cm} Nama: \makebox[3cm]{\hrulefill}\\ Pembimbing Tunggal
}

\end{document}

